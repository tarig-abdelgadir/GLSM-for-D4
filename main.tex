\documentclass{amsart}
\usepackage[utf8]{inputenc}

\usepackage{mymacros, stmaryrd, tikz, pstricks, pst-node, pst-text, pst-tree, subfigure, epsfig, psfrag}
\usepackage[all]{xy}
\usepackage{float}

\newtheorem{thm}{Theorem}
\newtheorem{cor}[thm]{Corollary}
\newtheorem{lem}[thm]{Lemma}
\newtheorem{prop}[thm]{Proposition}

\theoremstyle{definition}
\newtheorem{defn}[thm]{Definition}
\newtheorem{rmk}[thm]{Remark}
\newtheorem{notn}[thm]{Notation}
\newtheorem{conj}[thm]{Conjecture}
\newtheorem{eg}[thm]{Example}

\title{GLSM for D4}
\date{\today}

\begin{document}

\maketitle

\section{ICTP: January 2020}

The orbifold corresponding to the Dynkin diagram $D_4$ is given by the group 
\[
\mathbb{B}D_4:= \left\langle \sigma=\left(\begin{array}{cc} i & 0 \\ 0 & -i \end{array} \right), \tau=\left( \begin{array}{cc} 0 & 1 \\ -1 & 0 \end{array} \right) \right\rangle
\]
acting on $\mathbb{A}^2$.
The group $\bB D_4$ has five irreducible representations $\rho_0, \ldots, \rho_4$ whose ranks are $1,1,1,1,2$, respectively. The representation $\rho_0$ is the trivial representation.
%We fix a fibre functor $\Rep(BD_4) \rightarrow \textbf{Vect}$.
Below we express a GIT problem for which the stable quotient stack is $\bX:= [\bA^2/ \bB D_4]$ for one stability parameter and the minimal resolution $Y$ of the associated singularity for another.

We begin by setting up some notation.
We take $Q$ to be the McKay quiver of the action of $\bB D_4$ on $\bA^2$ and $A:= \bfk Q/I$ where $I$ is the corresponding ideal of relations.
We take $P_0, \ldots, P_4$ to be the projective left-modules of $A$ corresponding to the irreducible representations of $\bB D_4$.
Let $T_0, \ldots, T_4$ be vector spaces of dimensions $1,1,1,1,2$, respectively, and $G:= \text{GL}(\oplus T_i)$.
Fix an isomorphism $\bfk \simeq T_0$ once and for all; we will use this tacitly, replacing $T_0$ by $\bfk$ at times.
We write $L_4:= \det T_4$ and use $I^\dagger$ for the natural isomorphism $I^\dagger \colon T_4^\vee \otimes L_4 \rightarrow T_4$.

The GIT problem is given by the action of $G$ on the fields given below up to some constraints.
The fields are:
\begin{align}
&\bfk \simeq T_0 \\
\bfx \colon &T_0 \longrightarrow T_4 \label{fd:x} \\
\alpha \colon &T_1 \otimes T_1 \longrightarrow L_4 \label{fd:alpha} \\
\beta \colon &T_2 \otimes T_2 \longrightarrow L_4 \label{fd:beta} \\
\gamma \colon &T_3 \otimes T_3 \longrightarrow L_4 \label{fd:gamma} \\
\delta \colon &L_4 \otimes L_4 \longrightarrow T_1 \otimes T_2 \otimes T_3 \label{fd:delta} \\
P \colon &S^2(T_4) \longrightarrow T_1 \label{fd:P} \\ 
Q \colon &S^2(T_4) \longrightarrow T_2 \label{fd:Q} \\
R \colon &S^2(T_4) \longrightarrow T_3. \label{fd:R}
\end{align}

To express the constraints we need more notation.
Note that the field $P$, given in (\ref{fd:P}), maybe naturally rewritten as $P \colon T_4 \rightarrow T_4^\vee \otimes T_1$.
Composing with $I^\dagger$ and tensoring by an appropriate line bundles we get a morphism $$P^\dagger:= I^\dagger \circ P \colon T_4 \otimes L_4 \rightarrow T_4 \otimes T_1.$$
The fact that $P$ factors through the symmetric part $S^2(T_4)$ translates to tracelessness of $P^\dagger$.
We similarly define the morphisms $Q^\dagger$ and $R^\dagger$.

We may compose the fields $P^\dagger,Q^\dagger$ and $R^\dagger$ to express the constraints, they
are:
\begin{align}
    Q^\dagger R^\dagger &= R^\dagger Q^\dagger = \alpha \delta P^\dagger \label{con:qr} \\
    R^\dagger P^\dagger &= P^\dagger R^\dagger = \beta \delta Q^\dagger \label{con:rp} \\
    P^\dagger Q^\dagger &= Q^\dagger P^\dagger = \gamma \delta R^\dagger \label{con:pq} \\
    P^\dagger Q^\dagger R^\dagger &= \alpha \beta \gamma \delta^2 \text{Id}_{T_4} \label{con:pqr} \\
    \det P^\dagger &= \beta \gamma \delta^2 \\
    \det Q^\dagger &= \gamma \alpha \delta^2 \\
    \det R^\dagger &= \alpha \beta \delta^2.
\end{align}
We note that some of the constraints here are superfluous, like the determinant conditions for example; they are stated for convenience.

Fields satisfying the constraints gauged by $G$ above give us a quotient stack $\cM$.

\subsection{Tannakian duality}

Formally inverting the fields $\alpha, \ldots, R$ in (\ref{fd:alpha})-(\ref{fd:R}) gives a quotient stack $\widetilde{\cM}$ with an open embedding $\widetilde{\cM} \hookrightarrow \cM$.

\begin{proposition}\label{prop:tannaka}
The stack $\widetilde{\cM}$ is isomorphic to $\bX$.
\end{proposition}

\begin{proof}
To simplify notation we pick bases of $T_0, \ldots, T_4$ and let $G$ act by change of basis. 
One may gauge the scalar fields $\alpha, \ldots, \delta$ to $1$.
The subgroup of $G$ fixing the scalar fields at $1$ is $$\{(1,\lambda_1, \ldots, \lambda_4) \in G \,|\, \lambda_1^2=1, \lambda_2^2=1, \lambda_3 = \lambda_1\lambda_2, \det\lambda_4=1\} \subset G.$$

The field $P^\dagger$ is traceless and invertible so we may conjugate it to $\text{diag}(a, -a)$ for some $a \neq 0$.
We also have that $P^\dagger Q^\dagger$ is traceless which implies that $Q^\dagger$ is of the form $\sigma \cdot \text{diag}(b,c)$ where $b,c \neq 0$ and $\sigma$ is the nontrivial $2\times2$ permutation matrix.
The subgroup of the $\SL(4)$ component fixing $P^\dagger$ and $Q^\dagger$ up to scalar is now precisely $\bB D_4$ as presented above.
A careful calculation of how the scalars in $G$ interact with $\bB D_4 \subset \SL(T_4)$ reveals that they act as the rank 1 irreducible representations of $\bB D_4$.
\end{proof}

\begin{corollary}
Take $\phi$ to be the character of $G$ given by the weights $(5, -1,\ldots, -1)$ then $\cM^\phi \simeq \bX$
\end{corollary}

\begin{proof}
The determinant of any semi-invariants with weight $n \phi$, with $n>0$, must contain the scalar field $\alpha \beta \gamma \delta$ as a factor.
Therefore $\alpha \beta \gamma \delta \neq 0$ for any $\phi$-stable point of $\cM$ which implies that $\alpha, \ldots, R$ are invertible.  
\end{proof}

\subsection{The minimal resolution}

Let $\theta$ denote the character of $G$ with weights $-5,1,\ldots,1$ and $\cM^\theta$ denote the corresponding GIT quotient.
We will also use $\cN^\theta$ to denote the moduli space of quiver representations of the McKay quiver with stability parameter $\theta$.
Note that $\cN^\theta$ is isomorphic to $Y$.
We construct morphisms $f \colon \cN^\theta \rightarrow \cM^\theta$ and $g \colon \cM^\theta \rightarrow \cN^\theta$ and show that they are mutual inverses.

\subsection{The morphism $f \colon \cN^\theta \rightarrow \cM$}

%We start with the easier of the two.
Take a family of quiver representations $\cN^\theta(S)$ over a test scheme $S$.
Abusing notation, this gives vector bundles $T_0, \ldots, T_4$ over $S$ of the correct rank, as well as sections corresponding to the paths in the quiver.

\subsubsection{The field \bfx}

The arrow from vertex $0$ to vertex $4$ gives us a tautological section $T_0 \rightarrow T_4$ we take that to be the $\bfx$ of Equation~(\ref{fd:x}).

\subsubsection{The fields $\alpha, \beta, \gamma$}

Since our family is $\theta$-stable the universal section $T_4 \rightarrow T_1$ corresponding to the arrow $4 \rightarrow 1$ is surjective; its kernel is the line bundle $T_1^\vee \otimes L_4$.
The relations of the McKay quiver imply that the section corresponding to the path $1 \rightarrow 4 \rightarrow 1$ is zero and hence that the following is a complex of universal sections:
$$T_1 \rightarrow T_4 \rightarrow T_1 \rightarrow 0.$$
This gives us a section $T_1 \rightarrow T_1^\vee \otimes L_4$, in other words a morphism $T_1 \otimes T_1 \rightarrow L_4$ that we set to be $\alpha$ of Equation~(\ref{fd:alpha}).
We can pick the sections $\beta$ and $\gamma$ of Equations~\ref{fd:beta} and \ref{fd:gamma} in a similar fashion.

\subsubsection{The fields $P, Q, R$}

The construction of $P,Q$ and $R$ is more delicate.
The representations \underline{Hom}$(\rho_0, \rho_4)$ and \underline{Hom}$(\rho_4, \rho_1)$ are isomorphic.
This, in turn, gives us an isomorphism of corresponding projective modules of our quiver and further an isomorphism of vector spaces $\Hom(P_0, P_4) \simeq \Hom(P_4, P_1)$.
Now consider the composite $$T_0 \otimes_\bfk \Hom(P_0, P_4) \rightarrow T_0 \otimes_\bfk \Hom(P_4, P_1) \rightarrow T_4^\vee \otimes T_1$$ where the second factor is the evaluation map.
The stability condition $\theta$ implies $\text{ev} \colon T_0 \otimes_\bfk \Hom(P_0, P_4) \rightarrow T_4$ is surjective; checking that the kernel of ev is zero under the composite above gives us a morphism $T_4 \rightarrow T_4^\vee \otimes T_1$ that we will call $P$. 
The fields $Q$ and $R$ are constructed similarly.

\subsubsection{The constraints and the field $\delta$}

The space $\cM^\theta$ is birational to $\bX$ were the constraints hold; they hold on $\bX$ since the corresponding statements are true for representations of $\bB D_4$.

We therefore have, cf.\ (\ref{con:qr}), that the section $Q^\dagger R^\dagger$ differs from $P^\dagger$ by a constant on all of $\cN^\theta$ while $P^\dagger$ is nonzero.
Similar statements hold for $R^\dagger P^\dagger$ and $P^\dagger Q^\dagger$.
Stability of our family implies that $(\alpha, \beta, \gamma) \neq 0$, we may therefore read off the field $\delta$ from Equations~(\ref{con:qr}), (\ref{con:rp}), (\ref{con:pq}). 
This can probably be in a clearer fashion using complexes and quasi-isomorphisms but this will do for now.



\subsection{The morphism $g \colon \cM \rightarrow \cN$}

It suffices to define the sections corresponding to the arrows of the quiver and check that the relations are satisfied.

\subsubsection{The arrow $0 \rightarrow 4$}

For this we take the section $\bfx$.

\subsubsection{The arrow $4 \rightarrow 0$}
This is given by the composite
$$T_4 \xrightarrow{\text{id}\otimes u} T_4 \otimes T_4 \rightarrow L_4 \xrightarrow{\alpha\beta\gamma\delta^2} T_0.$$

\subsubsection{The other arrows $4 \rightarrow i$}
We construct this section for $i=1$, the others are similar.
The section corresponding to $1 \rightarrow 4$ is the following composite $$T_4 \xrightarrow{\text{id}\otimes u} T_4 \otimes T_4 \rightarrow S^2(T_4) \xrightarrow{P} T_1.$$

\subsubsection{The other arrows $i \rightarrow 4$}
Tensoring this section $T_4 \rightarrow T_1$ by $\alpha$ we get $T_4 \rightarrow T_1^\vee \otimes L_4$ this naturally gives a section $T_1 \rightarrow T_4^\vee \otimes L_4$, composing with $I^\dagger$ gives the desired section.

\subsubsection{The relations}

%The algebra $A$ is isomorphic to $\End_{\bX}(\oplus T_i)$.
The above gives us a homomorphism of algebras \begin{equation} \label{eq:kq1}
    \bfk Q \rightarrow \End_{\cM}(\oplus T_i).
\end{equation}
Proposition~\ref{prop:tannaka} gives an open embedding $\bX \hookrightarrow \cM$ through which pullback gives a monomorphism \begin{equation}\label{eq:kq2}
    \End_{\cM}(\oplus T_i) \hookrightarrow \End_{\bX}(\oplus T_i) \simeq A.
\end{equation}
The composite of the morphisms in (\ref{eq:kq1}) and (\ref{eq:kq2}) is a surjective epimorphism.
Putting these facts together we may conclude that $\End_{\cM}(\oplus T_i) \simeq A$.
That is, our homomorphism $\bfk Q \rightarrow \End_{\cM}(\oplus T_i)$ factors through $A$ and that our fields give sections that satisfy the relations of the quiver.

% \subsubsection{Stability}
% This follows from the isomorphism $\End_{\cM}(\oplus T_i) \simeq A$ discussed above.

% \subsubsection{The relation $i \rightarrow 4 \rightarrow i$}
% We do this for $i=0$ the others are similar.
% This follows since $$T_0 \xrightarrow{u} T_4 \xrightarrow{\text{id}\otimes u} T_4 \otimes T_4$$ factors through $S^2(T_4) \rightarrow T_4 \otimes T_4$.

% \subsubsection{The relation about the vertex 4}
% This is seemingly non-trivial, we need a lemma.
% We let $\bfx_i \colon T_0 \rightarrow T_4$ be the section corresponding to the path $0 \rightarrow 4 \rightarrow i \rightarrow 4$ for $i \neq 0$ and $\bfx_0:= \bfx$.

% \begin{lemma}
% For a family of $\theta$-stable representations of $\cM^\theta$ the following morphism is surjective
% $$T_0^{\oplus 4} \xrightarrow{(\bfx_0 \,\, \bfx_1 \,\, \bfx_2\,\, \bfx_3)} T_4.$$
% \end{lemma}

% \begin{proof}{\red [Rough argument]}
% We need to show that every morphism to $T_4$ factors through these.
% Now the first lemma gives us that we only need to look at the quiver to see all the paths from $T_0$ to $T_4$ and indeed these are the building blocks.
% \end{proof}

% Then we show that $P$ applied to these are the $\bfy$s

\subsection{The composite $g \circ f \colon \cN^\theta \rightarrow \cM \rightarrow \cN$ is the open inclusion}

It suffices to check this on the sections corresponding to each arrow.

\subsubsection{The arrow $0 \rightarrow 4$}
This is straight forward.

\subsubsection{The other arrows $4 \rightarrow i$}
We deal with case $i=1$ only; $i=2,3$ are similar.
Let $\bfe_\bfx \colon T_0 \rightarrow  T_0 \otimes \Hom(P_0, P_4)$ be the identity on the component corresponding to the arrow $0 \rightarrow 4$ and 0 elsewhere.
We then have the following commutative diagram:
\begin{equation}
    \xymatrix{T_0 \ar[r]^{\text{id}} \ar[d]^{\bfe_\bfx} & T_0 \ar[d]^\bfx \\
    T_0 \otimes \Hom(P_0, P_4) \ar[r]^-{\text{ev}} \ar[d]^P & T_4 \ar[d]^P \\
    T_0 \otimes \Hom(P_4, P_1) \ar[r]^-{\text{ev}} &T_4^\vee \otimes T_1.
    }
\end{equation}
Note that the field $P$ in the image is defined using the bottom half of the diagram.
The diagram combined with the fact that $P \colon \Hom(P_0, P_4) \rightarrow \Hom(P_4, P_1)$ takes $0 \rightarrow 4$ to $4 \rightarrow 1$ gives us that the section corresponding to $i \rightarrow 4$ gets mapped to itself under $g \circ f$.

\subsubsection{The arrows $i \rightarrow 4$}

For $i =1$, we may naturally rewrite the section corresponding the arrow $4 \rightarrow 1$ as a morphism $T_1^\vee \otimes L_4 \rightarrow T_4$ and fits into the short exact sequence in the diagram below:
\begin{equation}
    \xymatrix{ & T_1 \ar[r] \ar[d]^\alpha & 0 \ar[d] & \\
    0 \ar[r] & T_1^\vee \otimes L_4 \ar[r] & T_4 \ar[r] & T_1 \ar[r] & 0.} 
\end{equation}
Now since the field $\alpha$ in the image is given by the non-trivial vertical arrow we have the result for $i=1$.
The cases $i=2,3$ are similar.
The case $i=0$ follows from the relations of the quiver.

\section{Earlier correspondence}

We begin by introducing the GIT problem without much context just in case you came across something similar before.
We then hand-wave a little to explain the motivation behind this.

The orbifold corresponding to the Dynkin diagram $D_4$ is given by the group 
\[
\mathbb{B}D_4:= \left\langle \sigma=\left(\begin{array}{cc} i & 0 \\ 0 & -i \end{array} \right), \tau=\left( \begin{array}{cc} 0 & 1 \\ -1 & 0 \end{array} \right) \right\rangle
\]
acting on $\mathbb{C}^2$. 
The moduli space of $\theta$-stable quiver representation on the McKay quiver for any generic stability parameter $\theta$ will be denoted $\cM^\theta$.
We have that $\cM^\theta$ is isomorphic to the minimal resolution $Y$ of the corresponding singularity for any generic stability parameter $\theta$.

Take $T_1, \ldots, T_6$ to be vector spaces with dimension vector $(1,1,1,2,2,2)$.
Also impose the following restriction on the choice of vector spaces above.
In what follows $\text{det}$ is the top exterior power.
\begin{align*}
    \text{det}(T_5) &= T_1 \otimes T_2 \otimes T_3 \\
    \text{det}(T_6) &= \text{det}(T_4) \otimes T_1 \otimes T_2 \otimes T_3.
\end{align*}

The vector spaces $T_1, \ldots, T_4$ should be thought of as the fibres of the universal bundle over $\cM^\theta$ for some fixed $\theta$ (we take $T_0$ to be the $\bfk$).
In other words, the first chern classes of each these bundles are dual to the classes of the irreducible components in the exceptional locus of $Y$.
The spaces $T_5$ and $T_6$ are the fibres of rank two universal bundle arising from different choices of quiver stability parameters, up to tensoring by a universal line bundle.
Up to tensoring by a line bundle, $T_4$, $T_5$ and $T_6$ are the only rank two vector bundles on $Y$ arising as universal bundles on the $\cM^\theta$ for any generic quiver stability parameter.
For general $D_n$ the number of such rank two bundles is $2^{n-1}-n-1$.

For a reason that is not yet clear to me, the building blocks of the GIT problem that relate $Y$ to the stack $[\mathbb{C}^2/G]$ are the following fields:
\begin{align*}
&\bfk \longrightarrow T_4 \\
A \colon &T_5 \otimes T_1 \longrightarrow T_6 \\
B \colon &T_5 \otimes T_2 \longrightarrow T_6 \\
C \colon &T_5 \otimes T_3 \longrightarrow T_6 \\
M \colon &T_4 \otimes \text{det}(T_4) \longrightarrow T_6
\end{align*}
The morphisms $A$, $B$ and $C$ are required to satisfy the condition that $AB^{-1}$, $BC^{-1}$ and $CA^{-1}$ are traceless.
In particular, this is not really `linear'.
This condition of $A$, $B$, and $C$ comes from the mutual-commutativity of the corresponding elements $s_1$, $s_2$ and $s_3$ in the braid group.

Our gauge group is $$G := \{(g_1, \ldots,g_6) \in (\times_i \text{GL}(T_i)) \, |\, \text{det}(g_5)= g_1 g_2 g_3 \, , \,\, \text{det}(g_6)= \text{det}(g_4)g_1 g_2 g_3 \}.$$

We now show that we get the corresponding orbifold if we insist $A,B,C$ and $M$ are isomorphisms.
We may act by the element $$(1, \det M, 1, M, (\det A^{-1}\cdot \det M)^{\frac{1}{2}} \cdot A, \det M \cdot \text{Id}) \in G$$ to fix $M=\text{Id}$ and $A= \mu_1 \cdot \text{Id}$ for some scalar $\mu_1$. Therefore, we may assume $M = \text{Id}$ and $A = \mu_1 \cdot \text{Id}$.
The subgroup of $G$ fixing $M= \text{Id}$ and $A= \mu_1 \cdot \text{Id}$ is given by elements of the form $$(g_1, g_2, g_3, g_4, (g_1 g_2 g_3 \det g_4^{-1})^\frac{1}{2} \cdot g_4, (g_1g_2g_3)^\frac{1}{2} \cdot g_4) \in G$$

The tracelessness of $BA^{-1}$ and $CA^{-1}$ then implies that both $B$ and $C$ are traceless. We may act on $B$ by change of basis while preserving the fact that $M$ is the identity and $A$ is a scalar matrix. We may therefore assume that $B$ is of the form 
$$\begin{pmatrix}
\mu_2 & 0 \\
0 & - \mu_2
\end{pmatrix}.$$
Tracelness of $BC^{-1}$ then forces $C$ to be of the form 
$$\begin{pmatrix}
0 & \mu_3 \\
\mu_3 & 0
\end{pmatrix}.$$
To fix $\mu_1 = \mu_2 = \mu_3 = 1$ we act by $$(\mu_1, \mu_2, \mu_3, \text{Id}, (\mu_1\mu_2\mu_3)^\frac{1}{2}\cdot \text{Id}, (\mu_1\mu_2\mu_3)^\frac{1}{2}\cdot \text{Id}) \in G.$$ 
The result then follows from Tannakian duality.

\begin{rmk}
From this data we may construct a morphism $T_4 \otimes T_4 \rightarrow T_1$. First note that such a morphism is equivalent to a morphism $T_4 \rightarrow T_4^\vee \otimes T_1$. Invoking the natural isomorphism $T_4^\vee \cong T_4 \otimes \det(T_4)^\vee$, this is equivalent to a morphism $\det(T_4) \otimes T_4 \rightarrow T_4 \otimes T_1$.

Further applications of the natural isomorphism $V^\vee \cong V \otimes \det(V)^\vee$, gives us morphisms:
\begin{align*}
A^{T} &\colon T_6 \otimes T_1 \longrightarrow T_5 \otimes \det(T_4) \\
B^{T} &\colon T_6 \otimes T_2 \longrightarrow T_5 \otimes \det(T_4) \\
C^{T} &\colon T_6 \otimes T_3 \longrightarrow T_5 \otimes \det(T_4) \\
M^{T} &\colon T_6 \otimes \det(T_4) \longrightarrow T_4 \otimes T_1 \otimes T_2 \otimes T_3.
\end{align*}

Now the morphism $\det(T_4) \otimes T_4 \rightarrow T_4 \otimes T_1$ is given by the composite $MC^{T}BM$ (or equivalently $MB^{T}CM$): $$\det(T_4) \otimes T_4 \rightarrow T_6 \rightarrow T_5 \otimes \det(T_4) \otimes T_3^\vee \rightarrow T_6 \otimes \det(T_4) \otimes T_2^\vee \otimes T_3^\vee \rightarrow T_4 \otimes T_1.$$
\end{rmk}

\begin{rmk}
Note that one may reconstruct a representation of the McKay quiver equipped with this data along with a morphism $T_0 \rightarrow T_4$. Take, for example the arrow $T_4 \rightarrow T_1$, this is given by the composite $$T_4 \rightarrow T_4 \otimes T_4 \rightarrow T_1.$$
\end{rmk}

\begin{rmk}
Tracelessness of $AB^{-1}$ (and the like) follows from the need for  the equality $AB^{T}= \lambda BA^{T}$ for some scalar $\lambda$. That is, the morphisms $AB^{T}$ and $BA^{T}$ differ by tensoring by a line bundle. This comes from the braid relations.
\end{rmk}

\subsection*{Where does this come from?}

Any GLSM that recovers the orbifold should contain enough data to recover the stabiliser group at the origin. 
Tannakian philosophy implies that one should try to capture the monoidal data of the category of representations to achieve this result.

Our group $\bB D_4$ has four non-trivial irreducible representations: three of dimension 1 and a two dimensional one. We abuse notation by using the same letters $T_i$ to denote both the irreducile representations of $\bB D_4$ and the universal bundels on the corresponding McKay quiver.
We have that the monoidal structure is given by the following isomorphisms:
\begin{align*}
\text{ for } i \in \{1,2,3\} \quad T_i \otimes T_i &\rightarrow \det T_4 \\
\text{ for } \{i,j,k\} = \{1,2,3\} \quad T_i \otimes T_j &\rightarrow T_k\\
\det T_4  &\rightarrow T_0 \\
\text{S}^2(T_4) &\rightarrow T_1 \oplus T_2 \oplus T_3.
\end{align*}

We fix a generic stability parameter $\theta$ and $T_1, \ldots T_4$ be the universal bundles on $\cM^\theta$.
We look for generic stability parameter $\theta'$ for which the universal bundles are given by $(T_1 \otimes T_i), \ldots, (T_4 \otimes T_i)$ for $i = 1,2,3$ or $(T_1 \otimes \det T_4), \ldots, (T_4 \otimes \det T_4)$.
We then compute the element in the braid group taking the chamber containing $\theta$ to that containing $\theta'$.
This gives us a series of spherical twists each of which changes exactly one universal bundle.
The spherical twists also give us a morphism between the bundles across the corresponding hyperplanes.
The morphisms $A$, $B$, $C$ and $D$ above are such morphisms.
They correspond to the spherical twists corresponding to the simples at $1$, $2$, $3$ and $4$ respectively.
\end{document}
